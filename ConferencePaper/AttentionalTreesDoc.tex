\documentclass[10pt,a4paper]{article}
\usepackage[utf8]{inputenc}
\usepackage[english]{babel}
\usepackage{amsmath}
\usepackage{amsfonts}
\usepackage{amssymb}
\usepackage{graphicx}
\usepackage{lmodern}
\usepackage{fourier}
\usepackage[left=2cm,right=2cm,top=2cm,bottom=2cm]{geometry}
\author{Pilar de la Cruz}
\title{Attentional Trees in Robotics: towards an optimal executive attentional control meeting high-level decision making and control layer features.}
\begin{document}
\maketitle
\section*{Abstract}
\section*{Introduction}

 
\subsection*{Behavior Trees Pros}
\begin{itemize}
\item Ticking - the ability to tick allows for work between executions without multi-threading
\item Priority Handling - switching mechansims that allow higher priority interruptions is very natural
\item Simplicity - very few core components, making it easy for designers to work with it
\item Dynamic - change the graph on the fly, between ticks or from parent behaviours themselves
\end{itemize}
\section*{JSON Schema}
JSON is a standard format for data exchange inspired by JavaScript. Generally, JSON is in string or text format. A typical JSON schema consists of a list of string objects. The first object element defines the root of the Behavior Tree. The remaining object elements are defined as subtrees and will be added recursively as children from the root. Each object contains properties \verb|name|, \verb|type|, \verb|father|, \verb|children| and \verb|parameters|, which provide with information about the Behavior Tree node types as well as additional parameters for pre-and post-conditioning handling, priority execution and any additional information. This schema provides (1) simplicity due to its compact description of information, (2) flexibility due to its abstract base representation of tasks and (3) modularity due to the definition of subtrees which can be added and/orremoved on the fly in whatever position of Behavior Tree.


\begin{verbatim}
[
  {
    "name":"sequence01",
    "type":"sequence",
    "father": "",
    "children": [],
    "parameters": []
  },
  {
    "name":"selector01",
    "type":"selector",
    "father": "sequence01",
    "children": ["isReached","reach"],
    "parameters": ["distanceToObject", "emphasis", "reached"]
  },
  {
    "name":"selector02",
    "type":"selector",
    "father": "sequence01",
    "children": ["isPicked","pick"],
    "parameters": ["picked", "emphasis"]
  },
  {
    "name":"selector03",
    "type":"selector",
    "father": "sequence01",
    "children": ["isPlaced","place"],
    "parameters": ["placed", "emphasis"]
  }

]
\end{verbatim}




\subsection*{System Overview}
Attentional Trees framework aims at integrating the following features:
\begin{enumerate}
\item JSON schema parsing and Behavior Tree integration
\item Task planner based on the creation of the Behavior Tree
\item Priority handling of tasks based on the emphasis value
\item Event-based approach of Behavior Tree tick execution
\end{enumerate}
\subsubsection*{JSON Parsing}
\subsubsection*{Task planner}
\subsubsection*{Priority Handling}
\subsubsection*{Event-Based Behavior Tree}
\section*{Implementation}
The framework platform of Attentional Trees is based on the open-source py$\_$trees library. Py$\_$trees is compatible with ROS and provides the easiness and quickness of an implementation based in python language. Moreover, decoding operations becomes specially intuitive thanks to the in-built mapping between container and value (JSON objects to Python objects and viceversa) based on Python standard libraries.
\section*{Simulation Experiments}
In order to test the requirements described in section above, the following demo programs have been implemented\footnote{These are available on:}
\begin{itemize}
\item \verb|json-tree.py|. This program focuses on (1) loading JSON schema and (2) mapping into a Behavior Tree based on PA-BT approach \cite{b3}.
\item \verb|emphasis-tree.py|. This program focuses on high-priority handling based on access to emphasis value\cite{b1}.
\item \verb|attentional-tree.py|. This program focuses on event-based task execution of Behavior Trees. Tick signals are not periodically sent, but triggered by special events handled in the control layer.
\end{itemize}
\begin{thebibliography}{00}
\bibitem{b1} Add Matteo's paper and page of emphasis description
\bibitem{b2}https://www.guru99.com/python-json.html
\bibitem{b3} Behavior Tree book
\end{thebibliography}

\end{document}