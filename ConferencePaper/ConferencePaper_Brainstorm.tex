\documentclass[conference]{IEEEtran}
\IEEEoverridecommandlockouts
% The preceding line is only needed to identify funding in the first footnote. If that is unneeded, please comment it out.
\usepackage{cite}
\usepackage{amsmath,amssymb,amsfonts}
%\usepackage{algorithmic}
\usepackage{graphicx}
\usepackage{svg}
\usepackage{textcomp}
\usepackage{xcolor}
\def\BibTeX{{\rm B\kern-.05em{\sc i\kern-.025em b}\kern-.08em
    T\kern-.1667em\lower.7ex\hbox{E}\kern-.125emX}}
\begin{document}

\title{Attentional Trees And Robotics: Towards an Executive Framework Meeting High-level Decision Making and Control Layer Features.\\
}

\author{\IEEEauthorblockN{\textsuperscript{st} Pilar de la Cruz}
\IEEEauthorblockA{\textit{Intelligent And Interactive Systems Department} \\
\textit{University of Innsbruck}\\
Innsbruck, Austria \\
pilar.de-la-cruz@uibk.ac.at}
\and
\IEEEauthorblockN{\textsuperscript{nd} Matteo Saveriano}
\IEEEauthorblockA{\textit{Intelligent And Interactive Systems Department} \\
\textit{University of Innsbruck}\\
Innsbruck, Austria \\
matteo.saveriano@uibk.ac.at}
\and
\IEEEauthorblockN{\textsuperscript{nd} Justus Piater}
\IEEEauthorblockA{\textit{Intelligent And Interactive Systems Department} \\
\textit{University of Innsbruck}\\
Innsbruck, Austria \\
justus.piater@uibk.ac.at}
}

\maketitle

\begin{abstract}
(Brainstorm) Motivation: HRI. Attentional robot - flexible, if possible event-based. Introduce behavior tree trend. Potential in robotics, yet too high level. Merge. JSON. What is this system providing as new? Why is it convincing?
\end{abstract}

\begin{IEEEkeywords}
behavior trees, robotics, event-based, json
\end{IEEEkeywords}

\section{Introduction}
\subsection{Behavior Trees (Brainstorm)}
\begin{itemize}
\item Ticking - the ability to tick allows for work between executions without multi-threading
\item Priority Handling - switching mechansims that allow higher priority interruptions is very natural
\item Simplicity - very few core components, making it easy for designers to work with it
\item Dynamic - change the graph on the fly, between ticks or from parent behaviours themselves
\item Self-explainable
\end{itemize}
\subsection{Behavior Trees Limitations (Brainstorm)}
\subsection{How Limitations can be overcome: JSON (Brainstorm)}
JSON is a standard format for data exchange inspired by JavaScript. Generally, JSON is in string or text format. (EXTEND overview)
JSON schemas provide (1) simplicity due to its compact description of information, (2) flexibility due to its abstract base representation of tasks and (3) modularity due to the definition of subtrees which can be added and/or removed on the fly in whatever position of Behavior Tree.  Finally, JSON support with databsaes such as mySQL and MongoDB motivates the fact of bringing up a general string schema which couples not only task executions but also an abstract knowledge representation of actions.  
\subsection{What is this document about (Brainstorm)}
\subsection{Describe Sections Structure and Content}
\section{Related Works}
Add Matteo's system. Add SoA Behavior Trees.
\begin{itemize}
\item Matteo's system: Refer to Method's section. 
\end{itemize}

\section{System Overview}
Attentional Trees framework aims at integrating the following features:
\begin{enumerate}
\item Task planner based on Behavior Trees
\item Priority handling of tasks based on attentional mechanisms and emphasis \cite{b2}
\item JSON schema conversion into Behavior Trees 
%\item Event-based approach of Behavior Tree  execution
\end{enumerate}

\subsection{Task Planning}
\subsection{Priority Handling}
\subsubsection{Attentional Mechanism}
\subsubsection{Emphasis}
\subsection{JSON Schema}
 A typical JSON schema consists of a list of string objects. The first object element defines the root of the Behavior Tree. The remaining object elements are defined as subtrees and will be added recursively as children from the root. Each object contains properties \verb|name|, \verb|type|, \verb|father|, \verb|children| and \verb|parameters|, which provide with information about the Behavior Tree node types as well as additional attributes for pre-and post-conditioning handling, priority execution and any additional information. 
 
 \begin{verbatim}
{
    "name":"selector01",
    "type":"selector",
    "father": "sequence01",
    "children": ["isReached","reach"],
    "parameters": [["emphasis", 1]]
}


\end{verbatim}

\subsection{Event-Based Behavior Tree}

\section{Implementation}
The framework platform of Attentional Trees is based on the open-source library py$\_$trees \cite{b4}. py$\_$trees is compatible with ROS and provides the easiness and quickness of an implementation based in python language. Moreover, decoding operations becomes specially intuitive thanks to the in-built mapping between container and value (JSON objects to Python objects and viceversa) based on Python standard libraries.
*Note: justify the contribution of our work to the library 
\subsection{py tree Overview (Brainstorm)}
\begin{itemize}
\item idioms for PA-BT creation
\item blackboard and emphasis for attentional mechanism
\item trees and behavior classes
\item demo overviews

\end{itemize}
\subsection{Adapting Procedure (Brainstorm)}
\begin{itemize}
\item reasoning of applied demos as basis
\item use of idioms for PA-BT creation approach 
\item use of blackboard for emphasis parameter and priority handling
\item reasoning of behaviors' usage for simulations
\item read of schema variables and mapping for pre-condition, post-condition and blackboard variables

\end{itemize}


\section{Simulations}
In order to test the requirements described in section above, the following demo programs have been implemented\footnote{These are available on:}


\begin{itemize}
\item \verb|json-tree.py|. This program focuses on (1) loading JSON schema and (2) mapping into a Behavior Tree based on PA-BT approach \cite{b3}.
*Note: currently (8.12.2019) PA-BT is extended via \verb|automated-panning.py| demo. \verb|json-tree.py| provides with a Behavior Tree extension following a minimal implementation and no use of idioms.
%\begin{figure}
 % \includesvg{pa-bt}
 % \caption{Tree expansion rendered from json schema rendered in .dot graph}
  %\label{fig:json-tree}
%\end{figure}


%\begin{figure}
 % \includesvg{sequence01}
 % \caption{Tree expansion rendered from json schema rendered in .dot graph}
 % \label{fig:json-tree}
%\end{figure}
\item \verb|attentional-tree.py|. This program focuses on high-priority handling based on access to emphasis value\cite{b1}.

%\item \verb|automated-planning.py|. This program creates an extension of an behavior on the fly. It uses the idioms feature of library py$\_$tree. 
%\item \verb|attentional-tree.py|. This program focuses on event-based task execution of Behavior Trees. Tick signals are not periodically sent, but triggered by special events handled in the control layer.
\end{itemize}
\section*{Discussion}




\begin{thebibliography}{00}
\bibitem{b1} Add Matteo's paper and page of emphasis description
\bibitem{b2}https://www.guru99.com/python-json.html
\bibitem{b3} Behavior Tree book
\bibitem{b4} https://py-trees.readthedocs.io/en/devel/
\end{thebibliography}


\end{document}
